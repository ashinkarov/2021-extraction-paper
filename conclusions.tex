\section{\label{sec:concl}Conclusions and Future Work}


In this paper we investigate the idea of developing
embedded programs hand-in-hand with custom
code generators for them. We solve the well-known
conundrum of choosing between deep and shallow
embedding by leveraging the power of
reflection.
This allows us to enjoying the benefits
of shallow embedding, while keeping full access to the internal
structure of the embedded programs.

We apply this idea in the context of dependently-typed
languages to create verified implementations that can
be used in the context of an existing programming language.
We embed the target language in a theorem-prover, using
dependent types to encode the properties of interest.
Using reflection we bring the verified implementation
back into the original language.

We have demonstrated the approach by implementing
three embedded languages and two extractors, using
Agda as our host. Along the way we made some
improvements to the reflection API of Agda, and in the end we
used our embedding to implement (and verify) a practical
application --- a convolutional neural network.

The main advantages of our approach are twofold.  First,
our solution is fine-grained --- we can chose what part
of the application to embed, and what constructs of the
host language to extract.  Secondly, our extractors
can use the full power of dependent types to guarantee
safety of our embedded programs.

Right now we cannot yet guarantee that the
extracted code preservers the semantics of the original
implementation. While we rarely see
fully-verified compiler backends in the real world,
our approach is very close to enabling this.
We would need a formal semantics of the reflected language
and the proof that reflected programs respect it.
While this is non-trivial, a system like Agda could do
this in principle.

There is a number of improvements that can be added to
Agda and our extractors to make the resulting code more
efficient.  Supporting \AK{let}s in the internal syntax
would help to preserve sharing.  Recognising irrelevance
annotations in the extractors would help to eliminate
unused function arguments.  Introducing proper language
primitives to specify what exactly is an embedding would
be helpful. And finally, having access to more of
Agda's internals such as case trees would help to
generate more performant code.

Overall, this work only scratches the surface of extraction-based
compilation.  We never considered alternative theories supported
by theorem provers, \eg{} cubical type theory in Agda; we did not
consider recursive metaprogramming; we did not consider integrating
optimisations of extracted programs other than what rewriting rules
are capable to do.  All of these offer exciting research opportunities
on the way to make verified software easily accessible.
To paraphrase Jim Morrison: ``Safety and no surprise, The End''.
