\section{Conclusions and Future Work}
We have demonstrated\dots

\paragraph{Future Work}
Developments demonstrated in this paper open up exciting perspectives
for future research.
\begin{itemize}
  \item Do something with the semantics preservation.  We need to
    define semantics for Agda's internal syntax (or the subset
    that is being used to embed the DSL).  Then define the mapping
    from the target language syntax into Agda's internal syntax.
    Prove that extraction preserves the semantics for all the valid
    inputs.  In this case the only point of failure would be the
    mapping, which is much smaller than the entire extractor.
  \item Consider irrelevance annotations to function arguments.
    Agda makes it possible to annotate function arguments as irrelevant,
    meaning that it is guaranteed that they are only used during
    typechecking and may be erased at runtime.  Right now such annotations
    are ignored, but such erasures would benefit performance.
  \item Support let constructs in the internal syntax.  Currently
    let-s in Agda are treated purely as a syntactic sugar.  After
    desugaring all the lets are inlined, which makes the terms
    very large, and has a negative impact on performance when
    being executed without any optimisations.  Adding lets to the
    internal syntax would solve this problem, unfortunately this is
    far from trivial, as such a change generates enormous changes
    in the typechecker.
  \item Elegant extraction of recursive functions defined via
    well-founded recursion.
\end{itemize}

