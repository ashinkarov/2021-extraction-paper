%% For double-blind review submission, w/o CCS and ACM Reference (max submission space)
\documentclass[acmsmall,review,anonymous]{acmart}\settopmatter{printfolios=true,printccs=false,printacmref=false}
%% For double-blind review submission, w/ CCS and ACM Reference
%\documentclass[acmsmall,review,anonymous]{acmart}\settopmatter{printfolios=true}
%% For single-blind review submission, w/o CCS and ACM Reference (max submission space)
%\documentclass[acmsmall,review]{acmart}\settopmatter{printfolios=true,printccs=false,printacmref=false}
%% For single-blind review submission, w/ CCS and ACM Reference
%\documentclass[acmsmall,review]{acmart}\settopmatter{printfolios=true}
%% For final camera-ready submission, w/ required CCS and ACM Reference
%\documentclass[acmsmall]{acmart}\settopmatter{}


%% Journal information
%% Supplied to authors by publisher for camera-ready submission;
%% use defaults for review submission.
\acmJournal{PACMPL}
\acmVolume{1}
\acmNumber{CONF} % CONF = POPL or ICFP or OOPSLA
\acmArticle{1}
\acmYear{2018}
\acmMonth{1}
\acmDOI{} % \acmDOI{10.1145/nnnnnnn.nnnnnnn}
\startPage{1}

%% Copyright information
%% Supplied to authors (based on authors' rights management selection;
%% see authors.acm.org) by publisher for camera-ready submission;
%% use 'none' for review submission.
\setcopyright{none}
%\setcopyright{acmcopyright}
%\setcopyright{acmlicensed}
%\setcopyright{rightsretained}
%\copyrightyear{2018}           %% If different from \acmYear

%% Bibliography style
\bibliographystyle{ACM-Reference-Format}
%% Citation style
%% Note: author/year citations are required for papers published as an
%% issue of PACMPL.
\citestyle{acmauthoryear}   %% For author/year citations


%%%%%%%%%%%%%%%%%%%%%%%%%%%%%%%%%%%%%%%%%%%%%%%%%%%%%%%%%%%%%%%%%%%%%%
%% Note: Authors migrating a paper from PACMPL format to traditional
%% SIGPLAN proceedings format must update the '\documentclass' and
%% topmatter commands above; see 'acmart-sigplanproc-template.tex'.
%%%%%%%%%%%%%%%%%%%%%%%%%%%%%%%%%%%%%%%%%%%%%%%%%%%%%%%%%%%%%%%%%%%%%%


%% Some recommended packages.
\usepackage{booktabs}   %% For formal tables:
                        %% http://ctan.org/pkg/booktabs
\usepackage{subcaption} %% For complex figures with subfigures/subcaptions
                        %% http://ctan.org/pkg/subcaption

% Add definitions for abbreviations like e.g.; i.e; etc. with
% correct spacing depending on the parameters.  In this case
% `all' exposes all the definitions of the package, and `british'
% makes sure that there is no comma after e.g. or i.e.
\usepackage[all,british]{foreign}

\usepackage{fontspec}
\defaultfontfeatures{Scale=MatchLowercase}
\setmonofont{DejaVu Sans Mono}
%% poor man's solution for the DejaVu font:
%\setmonofont[Color={0019D4},ExternalLocation]{DejaVuSansMono.ttf}

% \usepackage{polyglossia}
% \setmainlanguage[spelling=new,latesthyphen=true]{UKenglish}
% %\hyphenpenalty=500 % default 50
% %\tolerance=400      % default 200


\newcommand*{\aplfont}{\defaultfontfeatures{Scale=MatchLowercase}%
\setmonofont[Color={0019D4}]{DejaVu Sans Mono}}
\newcommand{\apl}[1]{{\aplfont\texttt{#1}}}

\usepackage{mathpartir}

\usepackage{todonotes}

\usepackage{varwidth}

\usepackage{agda}
\usepackage{newunicodechar}
\newunicodechar{∷}{::}
\newunicodechar{→}{\ensuremath{\to}}
\newunicodechar{ω}{\ensuremath{\omega}}
\newunicodechar{⊎}{\ensuremath{\uplus}}
\newunicodechar{≔}{\ensuremath{\coloneqq}}
\newunicodechar{∎}{\ensuremath{\blacksquare}}
\newunicodechar{≟}{\ensuremath{\stackrel{?}{=}}}
\newunicodechar{ᵣ}{\ensuremath{_r}}
\newunicodechar{ₗ}{\ensuremath{_l}}
\newunicodechar{▴}{\ensuremath{\blacktriangle}}
\newunicodechar{▾}{\ensuremath{\blacktriangledown}}


% Some shortcut commands for agda symbols
\newcommand{\AD}[1]{\AgdaDatatype{#1}}
\newcommand{\AC}[1]{\AgdaInductiveConstructor{#1}}
\newcommand{\AF}[1]{\AgdaFunction{#1}}
\newcommand{\AB}[1]{\AgdaBound{#1}}
\newcommand{\AK}[1]{\AgdaKeyword{#1}}
\newcommand{\AR}[1]{\AgdaField{#1}}
\newcommand{\AS}[1]{\AgdaSymbol{#1}}
%\renewcommand{\AgdaCommentFontStyle}[1]{\textsf{#1}}

\newcommand\codeblock[1]{%
  %{\fbox{\begin{varwidth}{0.9\textwidth}#1\end{varwidth}}}
  {\begin{varwidth}{0.9\textwidth}#1\end{varwidth}}
}


% \usepackage{ifthen}
%
% % Insert extra space before some tokens.
% \DeclareRobustCommand{\AgdaFormat}[2]{%
%   \ifthenelse{
%     \equal{#1}{≡⟨} \OR
%     \equal{#1}{≡⟨⟩} \OR
%     \equal{#1}{∎}
%   }{\ }{}#2}


\begin{document}

%% Title information
\title[]{DSL Extraction in Agda}         %% [Short Title] is optional;
                                        %% when present, will be used in
                                        %% header instead of Full Title.
%\titlenote{with title note}             %% \titlenote is optional;
                                        %% can be repeated if necessary;
                                        %% contents suppressed with 'anonymous'
%\subtitle{Subtitle}                     %% \subtitle is optional
%\subtitlenote{with subtitle note}       %% \subtitlenote is optional;
                                        %% can be repeated if necessary;
                                        %% contents suppressed with 'anonymous'


%% Author information
%% Contents and number of authors suppressed with 'anonymous'.
%% Each author should be introduced by \author, followed by
%% \authornote (optional), \orcid (optional), \affiliation, and
%% \email.
%% An author may have multiple affiliations and/or emails; repeat the
%% appropriate command.
%% Many elements are not rendered, but should be provided for metadata
%% extraction tools.

%% Author with single affiliation.
\author{First1 Last1}
\authornote{with author1 note}          %% \authornote is optional;
                                        %% can be repeated if necessary
\orcid{nnnn-nnnn-nnnn-nnnn}             %% \orcid is optional
\affiliation{
  \position{Position1}
  \department{Department1}              %% \department is recommended
  \institution{Institution1}            %% \institution is required
  \streetaddress{Street1 Address1}
  \city{City1}
  \state{State1}
  \postcode{Post-Code1}
  \country{Country1}                    %% \country is recommended
}
\email{first1.last1@inst1.edu}          %% \email is recommended

%% Author with two affiliations and emails.
\author{First2 Last2}
\authornote{with author2 note}          %% \authornote is optional;
                                        %% can be repeated if necessary
\orcid{nnnn-nnnn-nnnn-nnnn}             %% \orcid is optional
\affiliation{
  \position{Position2a}
  \department{Department2a}             %% \department is recommended
  \institution{Institution2a}           %% \institution is required
  \streetaddress{Street2a Address2a}
  \city{City2a}
  \state{State2a}
  \postcode{Post-Code2a}
  \country{Country2a}                   %% \country is recommended
}
\email{first2.last2@inst2a.com}         %% \email is recommended
\affiliation{
  \position{Position2b}
  \department{Department2b}             %% \department is recommended
  \institution{Institution2b}           %% \institution is required
  \streetaddress{Street3b Address2b}
  \city{City2b}
  \state{State2b}
  \postcode{Post-Code2b}
  \country{Country2b}                   %% \country is recommended
}
\email{first2.last2@inst2b.org}         %% \email is recommended


%% Abstract
%% Note: \begin{abstract}...\end{abstract} environment must come
%% before \maketitle command
\begin{abstract}
  Most practical programming languages of today prioriterise 
  performance and usability over safety.  As a result, many
  useful correctness invariants cannot be expressed in the given
  language, typically due to weakness of the chosen type system.
  We are looking for non-invasive verification: how to
  guarantee a property about the given program without any modifications
  to the language toolchain.

  We study an approach when a programming language is shallowly
  embedded into a theorem prover, and dependent types are used
  to encode the properties of interest.   Bringing the embedded
  langauge and the properties back into the original language
  is the main focus of this work.  We propose a generic extraction
  mechanism that we implement in Agda that is based on the concept
  of reflection.  Making it possible to translate any shallowly
  embedded language in the target backend without the necessity
  to modify the theorem prover.
  We demonstrate our framework in action by encoding three languages:
  Kaleidoskope, SaC and a subset of APL, culminating in the implementation
  and extraction of a convolutional neural network.
\end{abstract}


%% 2012 ACM Computing Classification System (CSS) concepts
%% Generate at 'http://dl.acm.org/ccs/ccs.cfm'.
\begin{CCSXML}
<ccs2012>
<concept>
<concept_id>10011007.10011006.10011008</concept_id>
<concept_desc>Software and its engineering~General programming languages</concept_desc>
<concept_significance>500</concept_significance>
</concept>
<concept>
<concept_id>10003456.10003457.10003521.10003525</concept_id>
<concept_desc>Social and professional topics~History of programming languages</concept_desc>
<concept_significance>300</concept_significance>
</concept>
</ccs2012>
\end{CCSXML}

\ccsdesc[500]{Software and its engineering~General programming languages}
\ccsdesc[300]{Social and professional topics~History of programming languages}
%% End of generated code


%% Keywords
%% comma separated list
\keywords{keyword1, keyword2, keyword3}  %% \keywords are mandatory in final camera-ready submission


%% \maketitle
%% Note: \maketitle command must come after title commands, author
%% commands, abstract environment, Computing Classification System
%% environment and commands, and keywords command.
\maketitle

\section{Introduction}

It is very satisfying when a program comes with a formal proof of its
correct behaviour.  We can be confident that it implements the right algorithm,
and we can eliminate many subsequent checks if we use the prorgam as a building
block of a larger project.  Assume we wnat to provide such formal guarnatees
about some implementation --- how can we do this?

Unfortunately, most widely used programming languages of today are not equipped
with such capabilities.  For example, in C or Python, there is no way to give
a proof that a certain function never throws division by zero exceptions or
that it always indexes arrays within bounds.  Type systems of these languages
are to weak to support such invariants, as they exclusively focus on verifying
data domains.

% 
% The more correctness guarantees a programming language provides, the more
% difficult it is to use.  The reason is that a programmer has to do extra work
% to establish and maintain program invariants of interest.  Unfortunately, such
% extra work does not always pay off.  Correctness, at times, is orthogonal to
% functionality: if the program does not crash, it does not mean that it searches
% the given query on the internet.
% 
% As nobody seem to know how to write an invariant ``returns a meaningful
% response to a search query'', a lot of practical programming languages focus on
% usability, providing only rudimentary means for expressing correctness invariants.
% For example, C or Python are undoubtfully useful, but neither of the languages
% makes it possible to guarantee lack of division by zero or in-bound array
% indexing statically.
% 


While we can imagine extending the given type system, it is unlikely to work
well for an existing langauge.  Such an extension would touch on a very fundamental
design decisions of the language and likely to generate enormous changes in the
existing toolchains.  Adoption of such new features within the existing code
is also unclear.

In this paper we are investigating a less invasive approach on stating program
invariants that cannot be internalised in the given language.  We propose the
following three steps:
\begin{enumerate}
        \item encode the language (or its subset) in a theorem prover;
        \item use dependent types to formulate and prove required properties;
                and
        \item \emph{extract} the embedded program and verified properties back
                into the original language.
\end{enumerate}


We briefly summarise the main challenges associated with each step.

\paragraph{Embedding}
Encoding a target language within a host language is commonly referred as 
\emph{embedding}.
Two kinds of embeddings are commonly distinguished~\cite{}: deep and shallow.
Deep embedding encodes the target language AST and usually offers evaluation
function that maps expressions of the target language into expressions
of the host language.  In case of shallow embedding, each target language
construction gets a function of the host language.  As a result, we can encode
target language programs in the host language, but we do not necessarily get
a handle on the representation of these.  For more details refer to~\cite{}.

With deep embedding we have a full access to program representation, so it is
much easier to transform/manipulate embedded programs.  However, the type system
of the target language has to be encoded as well.  A number
of examples~\cite{} show us how advanced type systems such as
System-F~\cite{} or dependent types with families~\cite{} give rise to tagless 
interpreters~\cite{}, essentially allowing us to share the host type system for
typechecking target programs.  While this works beautifully for target languages with
simple type systems, the embedding gets quite ugly~\cite{} when the target type
system is dependently typed.  Impressive efforts have been taken~\cite{} to show
that the encoding is possible, but it is far from being practical.

In this work we consider shallow embedding only and we use Agda~\cite{} as our
host language.

\paragraph{Verifying} In the proposed system, dependent types play the key role
when it comes to verification.  First, we expect that all properties/invariants
about target programs are presented to the system as dependent types.  Note that
providing a proof of the properties is a job of a programmer --- dependently
typed system do not necessarily automate this step.  However, they do provide a
way to write the proof down, which in systems like Agda is not different from
writing a regular program~\cite{}.

Sometimes extrinsic verification~\cite{} of the target programs might be
insufficient, and we would have to use intrinsic approaches, \ie{} to interleave
the actual data and its constraints.  For example, ensuring that the function
$f(a, b) = a / (a - b)$ does not throw division by zero exception, we would have
to add the $a \nequiv b$ constraint to $f$.  Surely, all the functions that use
$f$ would have to respect this constraint.  In order to communicate this information,
it is important to ensure that our embedded language supports the notion of
dependent types.  Once again, that is where using shallow embedding helps
significantly.

\paragraph{Extracting}
One of the difficulties with the steps proposed so far lies in the inability
to use the verified target program outside of the theorem prover.  The key
contribution of this work is a solution to this problem.  We propose program extraction
technique that makes it possible to translate shallowly-embedded languages into
a target language of choice.  The extraction process is based on the reflection
capabilities~\cite{} of Agda, \ie{} a built-in metaprogramming facility that makes
it possible to obtain data representation of any given Agda term.  Esentially,
extraction process can be seen as a translation of one term representation
into another.

Specifying extraction of Agda terms in Agda offers a number of advantages.  First,
lack of external dependencies and full access to the extraction process.  Typically,
in classical compilers, backends are integral parts of the compiler, and while adding
a new backend or modifying existing one is perfectly possible, it is a non-trivial
endavour rarely accessible
from within the programming langauge.  Usually the only interface to the backend
we have is a set of predefined options.  Secondly, as extractor is a regular Agda
program, it can use all the advanced type system features of the langauge, so it
can encode as much properties about extraction correctness as needed.  Finally,
extraction plays very nicely with user-defined rewriting rules, which make it possible
to turn equality of two expressions into computation rule.  For example, while it
is possible to prove that double reverse of a list is identity, it is not a computation
rule.  By turning this into a rewriting rule, double reverses would be eliminated
from the terms prior extraction, resulting in a potentially better performing
target programs.
% TODO mention about turning proofs itno user-defined constraint or assertions 
%      depending on the capabilities of the backend language.


\paragraph{Applications} We demonstrate the propsed approach on two languages:
Kaleidoskope~\cite{} and SaC~\cite{}.  The former is a minimalist langauge
that is ofetn used as an example when creating LLVM~\cite{} backends.  We follow
the tradition and use it to demonstrate the basic mechanisms.  SaC is a high-performance
array language; it generates efficient code for various backends, but it is rather
restrictive.  The main focus lies in guarnteeng in-bound array indexing.  In order
to do so we have to track shapes of the arrays in types, which requires our
embeeded langauge to be dependently-typed.  Finally, we demonstrate an embedding
of a small subset of APL~\cite{}
that is sufficient to encode a Convolutional Neural Network (CNN)~\cite{}.
The main difficulty with APL is that the language is untyped, and the semantics
of the builtin operators is quite complex.  The main focus of the encoding is that
we define all the APL operators in terms of embeeded SaC operators, effectively
obtaining a compiler from a subset of APL to executable binaries.


\paragraph{Key Contributions}
The key contributions of this paper are:
\begin{itemize}
    \item We introduce the concept of extraction in Agda that is based on the
            reflection capabilities of the system and effectively demonstrate
            how to use dependently-typed meta-programming to create custom
            backends for embedded languages.
    \item We demonstrate how the proposed system can be applied, and we use
            two array languages as an example.  We start with a dependently-typed
            embedding of a subset of SaC~\cite{}.  We then define a subset of APL
            in terms of the first embedding.
    \item We demonstrate our system in action at the example of CNN implemented
            in APL.  We port the implementation into our system and ensure that
            it extracts correctly.
\end{itemize}

The rest of the paper is organised as follows:
\todo[inline]{FIXME}

\input{latex/background}
\input{latex/kaleidoskope}
\input{latex/arraylang}
\input{latex/aplcnn}
\section{Related Work}


\section{\label{sec:concl}Conclusions and Future Work}


In this paper we investigate the notion of extraction.  This
is the ability to express programs hand-in-hand with custom
code generators for them.  We have demonstrated that
compile-time metaprogramming is sufficient to equip
a programming language with extraction capabilities.
A well-known conundrum of choosing between deep and
shallow embedding can be resolved if the host language
is extraction-capable.  While enjoying the benefits
of shallow embedding, there is a full access to the internal
structure of the embedded programs.

We use this idea in the context of dependently-typed
languages to create verified implementations that can
be used in the context of an existing programming language.
We embed a language of interest in a theorem-prover, using
dependent types to encode the properties of interest.
Using extraction we bring the verified implementation
back into the original language.

We have demonstrated the approach by implementing
three embedded languages and two extractors using
Agda as our host.  We have demonstrated the
changes to Agda that we had to do on the way, and we
used our embedding to implement (and verify) a practical
application --- a convolutional neural network.

The main advantages are twofold.  First,
our solution is fine-grain --- we can chose what part
of the application to embed, and what constructs of the
host language to extract.  Secondly, extractors
can use the full power of dependent types guaranteeing
some safety properties of interest.

In order to make the proposed idea even more powerful,
we would require improvements to metaprogramming capabilities
of theorem provers.

Most importantly, right now it is not guaranteed that the
extracted code preservers the semantics of the original
implementation.  Essentially, one has to trust that the
extractor is doing the right job.  While we rarely see
fully-verified compiler backends in the real world, in
the proposed environment we are very close to enabling this.
We would need a formal semantics of the reflected language
and the proof that reflected programs respect it.
While this is non-trivial, a system like Agda could do
this in principle.

There is a number of improvements that can be added to
Agda and our extractors to make the resulting code more
efficient.  Supporting `let's in the internal syntax
would help to preserve sharing.  Recognising irrelevance
annotations in the extractors would help to eliminate
unused function arguments.  Introducing proper language
primitives to specify what exactly is an embedding would
be helpful.

\todo[inline]{Mention improving performance, mention the we
  can extract agda in agda and do some jitting}

Overall, this work only scratches the surface of extraction-based
compilation.  We never considered alternative theories supported
by theorem provers, \eg{} cubical type theory in Agda; we did not
consider recursive metaprogramming; we did not consider integrating
optimisations of extracted programs besides of what rewriting rules
are capable to do.  All of these offer exciting research opportunities
on the way to make verified software easily accessible.
To paraphrase Jim Morrison: ``Safety and no surprise, The End''.





%% Acknowledgments
\begin{acks}                            %% acks environment is optional
                                        %% contents suppressed with 'anonymous'
  %% Commands \grantsponsor{<sponsorID>}{<name>}{<url>} and
  %% \grantnum[<url>]{<sponsorID>}{<number>} should be used to
  %% acknowledge financial support and will be used by metadata
  %% extraction tools.
  This material is based upon work supported by the
  \grantsponsor{GS100000001}{National Science
    Foundation}{http://dx.doi.org/10.13039/100000001} under Grant
  No.~\grantnum{GS100000001}{nnnnnnn} and Grant
  No.~\grantnum{GS100000001}{mmmmmmm}.  Any opinions, findings, and
  conclusions or recommendations expressed in this material are those
  of the author and do not necessarily reflect the views of the
  National Science Foundation.
\end{acks}


%% Bibliography
\bibliography{paper}


\end{document}
